\documentclass[11pt,twocolumn,a4paper]{article}
\usepackage[top=1.4cm,bottom=1cm,left=1cm,right=1cm]{geometry}
												%設定留白
\usepackage{fontspec}							%設定字體
\usepackage{color}
\usepackage{xeCJK}								%xeCJK
\usepackage{listings}							%顯示code用的
\usepackage[Sonny]{fncychap}					%排版,頁面模板
\usepackage{fancyhdr}							%設定頁首頁尾
\usepackage[compact]{titlesec}  				%tielespacing
\usepackage{enumerate}							%ordered item
\usepackage[bookmarks,hidelinks]{hyperref}		%make bookmarks

%\topmargin=0pt
\headsep=5pt
%\textheight=750pt
\footskip=20pt
\columnsep=5pt									%兩欄間隔
%\voffset=-40pt
%\textwidth=520pt
%\marginparsep=0pt
%\marginparwidth=0pt
%\marginparpush=0pt
%\oddsidemargin=0pt
%\evensidemargin=0pt
%\hoffset=-30pt

\title{Codebook v1.3}
\author{kevinptt}
\date{May 16, 2014}
\setmainfont{Perpetua}							%主要字型
\setmonofont{Consolas}							%等寬字型
\setCJKmainfont{文泉驛微米黑}					%中文字型
\XeTeXlinebreaklocale "zh"						%中文自動換行
\titleformat{\section}{\normalfont\LARGE\bfseries}{\thesection}{1em}{}
\titlespacing{\section}{0pt}{0pt}{0pt}
\titlespacing{\subsection}{0pt}{-20pt}{0pt}

%%%%%%%%%%%%%%%%%%%%%%%%%%%%%%

\lstset{										% Code顯示
language=C++,									% the language of the code
basicstyle=\footnotesize\ttfamily, 				% the size of the fonts that are used for the code
numbers=none,									% where to put the line-numbers
numberstyle=\ttfamily,							% the size of the fonts that are used for the line-numbers
stepnumber=1,									% the step between two line-numbers. If it's 1, each line will be numbered
numbersep=8pt,									% how far the line-numbers are from the code
backgroundcolor=\color{white},					% choose the background color. You must add \usepackage{color}
showspaces=false,								% show spaces adding particular underscores
showstringspaces=false,							% underline spaces within strings
showtabs=false,									% show tabs within strings adding particular underscores
frame=no,										% adds a frame around the code
tabsize=2,										% sets default tabsize to 2 spaces
captionpos=b,									% sets the caption-position to bottom
breaklines=true,								% sets automatic line breaking
breakatwhitespace=false,						% sets if automatic breaks should only happen at whitespace
title=\lstname,									% show the filename of files included with \lstinputlisting; also try caption instead of title
escapeinside={\%*}{*)},							% if you want to add a comment within your code
morekeywords={*}								% if you want to add more keywords to the set
}

%%%%%%%%%%%%%%%%%%%%%%%%%%%%%%

\begin{document}
%\pagestyle{fancy}
%\fancyfoot{}
%\fancyhead{}
%\fancyhead[R]{\thepage}
%\renewcommand{\headrulewidth}{0.4pt}
%\renewcommand{\footrulewidth}{0pt}
\renewcommand{\contentsname}{Index}
\tableofcontents

%%%%%%%%%%%%%%%%%%%%%%%%%%%%%%
\newpage
\section{Enviroment Settings}
\subsection{.vimrc}
\begin{lstlisting}[label=.vimrc,language=bash]
" set encoding
set encoding=utf-8
set fileencodings=utf-8,big5
set showmode
syntax on
set hlsearch
set background=dark
set laststatus=2
set wildmenu
set scrolloff=5 " keep at least 5 lines above/below
set ruler
set cursorline
set ic    " ignore case when searching
set bs=2  " enable backspace
set number
set tabstop=4
set shiftwidth=4
set autoindent
set smarttab
set smartindent
map<F9> :!g++ "%:t" -o "%:r.out" -Wall -Wshadow -O2 -Im && "./%:r.out"
map<F10> :!g++ "%:t" -o "%:r.out" -Wall -Wshadow -O2 -Im
\end{lstlisting}

\subsection{.screenrc}
\begin{lstlisting}[label=.screenrc,language=bash]
screen -t home 0
screen -t pA 1
screen -t pB 2
# Look and feel
caption always "%{WK}Screen: %n | %h %=%t %Y/%m/%d %c:%s"
hardstatus alwayslastline "%-Lw%{= BW}%50>%n%f* %t%{-}%+Lw%<"
256-Color
# Always start screen with utf8 enabled. (Alternative method is to run screen with -U parameter)
defutf8 off
# Left Right meta key mods
#ALT-<> and C-Left C-Right
bindkey \033[D prev
bindkey \033[C next
bindkey "^[," prev
bindkey "^[." next
\end{lstlisting}

%%%%%%%%%%%%%%%%%%%%%%%%%%%%%%
\newpage
\section{Data Structure}
\subsection{Fenwick Tree [1, size]}
\begin{lstlisting}[label=Fenwick Tree]
inline int lowbit(int x) { return x&-x; }
template<class T>
class fenwick {
public:
	fenwick(int __size=SIZE) {
		size = __size+10;
		a = new T[size], b=new T[size];
		memset(a, 0, sizeof(T)*size);
		memset(b, 0, sizeof(T)*size);
	}
	~fenwick() { delete[] a, delete[] b;}
	inline void add(int l, int r, long long n) {

	__add(a, r, r*n), __add(a, l-1, (l-1)*-n);
		__add(b, r, n), __add(b, l-1, -n);
	}
	inline long long sum(int l, int r) { return __sum(r)-__sum(l-1); }
private:
	int size;
	T *a, *b;
	inline void __add(T *arr, int x, T n) { for(; x&&n&&x<size; x+=lowbit(x)) arr[x]+=n; }
	inline T __sum(T x) { return __sum(a, x)+(__sum(b, size)-__sum(b, x))*x; }
	inline T __sum(T *arr, int x) {
		T res=0;
		for(; x; x-=lowbit(x)) res+=arr[x];
		return res;
	}
};
\end{lstlisting}

\subsection{Fenwick Tree 2D - [1, size][1, size]}
\begin{lstlisting}[label=Fenwick Tree 2D]
int tree[size+1][size+1]={{0}};
inline int lowbit(const int &x) {return x&(-x);}
inline void add(int x, int y, int z) {
	for(int i; x<=n; x+=lowbit(x))
		for(i=y; i<=n; i+=lowbit(i)) tree[x][i]+=z;
}
inline int query(short x, short y) {
	int res=0;
	for(int i; x; x-=lowbit(x))
		for(i=y; i; i-=lowbit(i))
			res+=tree[x][i];
	return res;
}
\end{lstlisting}

\newpage
\subsection{Heap}
\begin{lstlisting}[label=Heap]
// max heap tree
#define ParentIndex(i) i==0 ? 0 : ((i-1) >> 1)
#define LeftChildIndex(i) ((i)<<1)+1
#define RightChildIndex(i) ((i)<<1)+2 
void BuildMaxHeap(int*, const int&);
void MaxHeapBalance(int*, const int&, const int&);
void MaxHeapDelete(int*, int&); 
inline bool comp(int &a, int &b) {return a>b;}
void BuildMaxHeap(int all[], const int &size) {
	for(int i=(size-1) >> 1; i>=0; i--)
		MaxHeapBalance(all, size, i);
}
void MaxHeapBalance(int all[], const int &size, const int &root) {
	int aim = root, aim2;
	while(1) {
		aim2 = aim;
		int L = LeftChildIndex(aim2);
		int R = RightChildIndex(aim2);
		if( L < size && comp( all[aim], all[L] ) ) aim = L;
		if( R < size && comp( all[aim], all[R] ) ) aim = R;
		if( aim != aim2 ) swap(all[aim], all[aim2]);
		else return;
	}
}
void MaxHeapAdd(int all[], int &size, const int &AddNum) {
	all[size] = AddNum;
	++size;
	int P, index = size-1;
	while( index != 0 ) {
		P = ParentIndex(index);
		if( comp(all[P], all[index]) ) {
			swap(all[P], all[index]);
			index = P;
		}
		else return;
	}
}
void MaxHeapDelete(int all[], int &size) {
	all[0] = all[size-1], --size;
	MaxHeapBalance(all, size, 0);
}
\end{lstlisting}

\newpage
\subsection{Deap}
\begin{lstlisting}[label=Deap]
class deap {
public:
	deap() {size=1;}
	~deap() {}
	inline void insert(int n) {
		arr[++size]=n;
		int now=size;
		if( (now&1) && arr[now-1]>arr[now] )
			swap(arr[now-1], arr[now]), now--;
		while( now>3 ) {
			if( arr[now>>2<<1]>arr[now] )
				swap(arr[now>>2<<1], arr[now]),
				now=now>>2<<1;
			else if( arr[(now>>2<<1)+1]<arr[now] )
				swap(arr[(now>>2<<1)+1], arr[now]),
				now=(now>>2<<1)+1;
			else break;
		}
	}
	inline int min() {
		int res=arr[2];
		swap(arr[2], arr[size--]), down(2);
		return res;
	}
	inline int max() {
		int res=arr[3];
		swap(arr[3], arr[size--]), down(3);
		return res;
	}
private:
	int arr[1000005], size;
	inline void down(int now) {
		while( (now<<1)<=size ) {
			int tmp;
			if( (now&1)==0 ) {
				if( arr[now]>arr[now+1] )
					{swap(arr[now], arr[now+1]);
					now++;continue;}
				tmp=now;
				if( arr[tmp]>arr[now<<1] )
					tmp=now<<1;
				if( (now<<1)+2<=size && arr[tmp]>arr[(now<<1)+2] ) tmp=(now<<1)+2;
				if( tmp==now ) break;
				else swap(arr[now], arr[tmp]),
					now=tmp;
			}
			else if( (now&1)==1 ) {
				if( arr[now]<arr[now-1] )
					{swap(arr[now], arr[now-1]);
					now--;continue;}
				tmp=now;	
				if( arr[tmp]<arr[(now<<1)-1] )
					tmp=(now<<1)-1;
				if( (now<<1)+1<=size && arr[tmp]<arr[(now<<1)+1] ) tmp=(now<<1)+1;
				if( tmp==now ) break;
				else swap(arr[now], arr[tmp]), now=tmp;
			}
		}
		if( (now&1)==0 && now+1<=size && arr[now]>arr[now+1] )
			swap(arr[now], arr[now+1]);
		if( (now&1)==1 && arr[now]<arr[now-1] )
			swap(arr[now], arr[now-1]);
	}
};
\end{lstlisting}

\subsection{zkw Segment Tree\\(range modify and query)}
\begin{lstlisting}[label=zkw Segment Tree]
class zkw_seg_tree { public:
	struct node {
		node() {add=sum=0, len=1;}
		int len, add, sum; 
	};
	zkw_seg_tree(int size) { // [1,size]
		dep=lg2(size-1)+1;
		delta=(1<<dep)-1;
		arr=new node[1<<(dep+1)];
		for(int i=delta; i>0; --i) arr[i].len=arr[i+i].len<<1;
	}
	~zkw_seg_tree() {delete[] arr;}
	inline void update(int l, int r, int num=1) {
		l+=delta-1, r+=delta+1;
		int l0=l, r0=r;
		while( r-l>1 ) {
			if( (l&1)^1 ) ++l, arr[l].add+=num, arr[l].sum+=arr[l].len*num;
			if( (r&1)^0 ) --r, arr[r].add+=num, arr[r].sum+=arr[r].len*num;
			l>>=1, r>>=1;
		}
		__update(l0), __update(r0);
	}
	inline int query(int l, int r) {
		__down(l+delta), __down(r+delta);
		l+=delta-1, r+=delta+1;
		int res=0;
		while( r-l>1 ) {
			if( (l&1)^1 ) res+=arr[l+1].sum;
			if( (r&1)^0 ) res+=arr[r-1].sum;
			l>>=1, r>>=1;
		}
		return res;
	}
private:
	node *arr;
	int dep, delta;
	inline int lg2(int x) {int r;for(r=-1; x; x>>=1, ++r);return r;}
	inline void __update(int x) {
		while( x>1 ) x>>=1, arr[x].sum=arr[x+x].sum+arr[x+x+1].sum+arr[x].len+arr[x].add;
	}
	inline void __down(int x) {
		for(int i=dep, tmp; i>0; --i) {
			tmp=x>>i;
			arr[tmp<<1].add+=arr[tmp].add;
			arr[(tmp<<1)+1].add+=arr[tmp].add;
			arr[tmp<<1].sum+=arr[tmp].add*arr[tmp<<1].len;
			arr[(tmp<<1)+1].sum+=arr[tmp].add*arr[tmp<<1].len;
			arr[tmp].add=0;
		}
	}
} segtree(N);
\end{lstlisting}

\newpage
\subsection{劃分樹}
\begin{lstlisting}[label=劃分樹]
#include <iostream>
#include <cstdio>
#include <algorithm>
using namespace std;
#define N 100005
int a[N], as[N];//原數組,排序後數組 
int n, m;
int sum[20][N];//紀錄第i層的1~j劃分到左子樹的元素個數(包括j)
int tree[20][N];//紀錄第i層元素序列
void build(int c, int l, int r) {
	int i, mid=(l+r)>>1, lm=mid-l+1, lp=l, rp=mid+1;
	for (i=l; i<=mid; i++)
		if (as[i] < as[mid]) lm--;
			//先假設左邊的(mid-l+1)個數都等于as[mid],然后把實際上小于as[mid]的減去
	for (i = l; i <= r; i++){
		if (i == l) sum[c][i] = 0;
			//sum[i]表示[l, i]內有多少個數分到左邊,用DP來維護 
		else sum[c][i] = sum[c][i-1];
		if (tree[c][i] == as[mid]){
			if (lm){
				lm--;
				sum[c][i]++;
				tree[c+1][lp++] = tree[c][i];
			}else
				tree[c+1][rp++] = tree[c][i];
		} else if (tree[c][i] < as[mid]){
			sum[c][i]++;
			tree[c+1][lp++] = tree[c][i];
		} else
			tree[c+1][rp++] = tree[c][i];
	}
	if (l == r)return;
	build(c+1, l, mid);
	build(c+1, mid+1, r);
}
int query(int c, int l, int r, int ql, int qr, int k){
	int s;//[l, ql)內將被劃分到左子樹的元素數目
	int ss;//[ql, qr]內將被劃分到左子數的元素數目
	int mid=(l+r)>>1;
	if (l == r)
		return tree[c][l];
	if (l == ql){//這裡要特殊處理!
		s = 0;
		ss = sum[c][qr];
	}else{
		s = sum[c][ql 1];
		ss = sum[c][qr]- ;
	} //假設要在區間[l,r]中查找第k大元素,t為當前節點,lch,rch為左右孩子,left,mid為節點t左邊界界和中間點。
	if (k <= ss)//sum[r]-sum[l-1]>=k,查找lch[t],區間對應為[ left+sum[l-1], left+sum[r]-1 ]
		return query(c+1, l, mid, l+s, l+s+ss-1, k);
	else
		//sum[r]-sum[l-1]<k,查找rch[t],區間對應為
		[mid+1+l-left-sum[l-1], mid+1+r-left-sum[r]]
		return query(c+1, mid+1, r, mid-l+1+ql-s, mid-l+1+qr-s-ss, k-ss);
}
int main(){
	int i, j, k;
	while(~scanf("%d%d", &n, &m)){
		for(i=1; i<=n; i++) {
			scanf("%d", &a[i]);
			tree[0][i] = as[i] = a[i];
		}
		sort(as+1, as+1+n);
		build(0, 1, n);
		while(m--){
			scanf("%d%d%d", &i, &j, &k);
				// i,j分別為區間起始點,k為該區間第k大的數。
			printf("%d\n", query(0, 1, n, i, j, k));
		}
	}
	return 0;
}
\end{lstlisting}

\newpage
\subsection{BigNum}
\begin{lstlisting}[label=BigNum]
#include <cstdio>
#include <cstring>
template<class T>
T max(const T& a, const T& b) {return a>b?a:b;}
template<class T>
T abs(const T& n) {return n>=T(0)?n:-n;}
class BigNum {
public:
	BigNum(const int& num=0) : len(0), sign(1) {
		int num2=num;
		memset(arr, 0, sizeof(arr));
		if( num2<0 ) sign=-1, num2*=-1;
		while( num2 ) arr[len++]=num2%step, num2/=step;
	}
	BigNum(const char* num0) : len(0), sign(1) {
		*this = num0;
	}
	BigNum(const BigNum& b) : len(b.len), sign(b.sign) {
		memset(arr, 0, sizeof(arr));
		for(int i=0; i<len; ++i) arr[i]=b.arr[i];
	}
	~BigNum() {}
	BigNum & operator=(const BigNum& b) {
		len=b.len;
		sign=b.sign;
		memset(arr, 0, sizeof(arr));
		for(int i=0; i<len; ++i) arr[i]=b.arr[i];
		return *this;
	}
	BigNum & operator=(const int& num) {
		int num2=num;
		memset(arr, 0, sizeof(arr));
		len=0, sign=1;
		if( num2<0 ) sign=-1, num2*=-1;
		while( num2 ) arr[len++]=num2%step, num2/=step;
		return *this;
	}
	BigNum & operator=(const char* num0) {
		char num[strlen(num0)];
		int offset = 0;
		if( num0[0] == '-' ) sign = -1, ++offset;
		while( num0[offset]=='0' ) ++offset;
		strcpy(num, num0+offset);
		int tmp = strlen(num);
		for(int i=tmp-digit; i>=0; i-=digit) {
			arr[len] = 0;
			for(int j=0; j<digit; ++j) arr[len] = arr[len]*10 + num[i+j]-'0';
			++len;
		}
		arr[len] = 0;
		for(int j=0; j<tmp%digit; ++j) arr[len] = arr[len]*10 + num[j]-'0';
		++len;
		return *this;
	}
	BigNum operator+(const BigNum& b) const {
		if( *this>0 && b<0 ) return *this-(-b);
		if( *this<0 && b>0 ) return -(-*this-b);
		BigNum res=*this;
		int len2=max(res.len, b.len);
		for(int i=0; i<len2; ++i) {
			res.arr[i]+=b.arr[i];
			if( res.arr[i]>=step ) res.arr[i]-=step, res.arr[i+1]++;
		}
		res.len=len2;
		if(res.arr[len2]) ++res.len;
		return res;
	}
	BigNum operator-(const BigNum& b) const {
		if( *this<b ) return -(b-*this);
		if( *this<0 && b<0 ) return -(-*this+b);
		if( *this>0 && b<0 ) return *this+(-b);
		BigNum res=*this;
		int len2=max(res.len, b.len);
		for(int i=0; i<len2; ++i) {
			res.arr[i]-=b.arr[i];
			if( res.arr[i]<0 ) res.arr[i]+=step, res.arr[i+1]--;
		}
		while( len2>0 && res.arr[len2-1]==0 ) --len2;
		res.len=len2;
		return res;
	}
	BigNum operator*(const BigNum& b) const {
		if( *this==0 || b==0 ) return BigNum(0);
		BigNum res;
		for(int i=0; i<len; ++i) {
			for(int j=0; j<b.len; ++j) {
				res.arr[i+j]+=arr[i]*b.arr[j];
				res.arr[i+j+1]+=res.arr[i+j]/step;
				res.arr[i+j]%=step;
			}
		}
		res.len=len+b.len-1;
		while( res.arr[res.len] ) ++res.len;
		res.sign=sign*b.sign;
		return res;
	}
	BigNum operator/(const int& b) const {
		if( b==0 ) return 0;
		BigNum res;
		long long reduce=0;
		int signb=b>0?1:-1, b2=b*signb;
		for(int i=len-1; i>=0; --i) {
			res.arr[i] = (arr[i]+reduce*step)/b2;
			reduce = (arr[i]+reduce*step)%b2;
		}
		res.len = len;
		while( res.len>0 && res.arr[res.len-1]==0 ) --res.len;
		if( res.len==0 ) res.sign=1;
		else res.sign=sign*signb;
		return res;
	}
	BigNum operator/(const BigNum& b) const {
		BigNum abs_this=abs(*this);
		if( b==0 ) return 0;
		BigNum st=0, ed, md;
		if( b.arr[0]>0 ) ed=abs_this/b.arr[0];
		else if( b.arr[1]*b.step+b.arr[0]>0 ) ed=abs_this/b.arr[1]*b.step+b.arr[0];
		else ed=abs_this;
		while( st<ed ) {
			md = (st+ed)/2+1;
			if( md*b<=abs_this ) st=md;
			else ed=md-1;
		}
		if( st.len==0 ) st.sign=1;
		else st.sign=sign*b.sign;
		
		return st;
	}
	BigNum operator%(const int& b) const {
		if( b<=0 ) return 0;
		BigNum res;
		long long reduce=0;
		for(int i=len-1; i>=0; --i)
			reduce = (arr[i]+reduce*step)%b;
		return reduce*sign;
	}
	BigNum operator%(const BigNum& b) const {
		if( b.isInt() ) return *this%int(b.toInt());
		if( b<=0 ) return 0;
		return *this-*this/b*b;
	}
	bool operator<(const BigNum& b) const {
		if( sign!=b.sign ) return sign<b.sign;
		if( len!=b.len ) return len*sign<b.len*b.sign;
		for(int i=len-1; i>=0; --i)
			if( arr[i]!=b.arr[i] ) return arr[i]*sign<b.arr[i]*b.sign;
		return false;
	}
	bool operator==(const BigNum& b) const {
		if( sign!=b.sign ) return false;
		if( len!=b.len ) return false;
		for(int i=len-1; i>=0; --i)
			if( arr[i]!=b.arr[i] ) return false;
		return true;
	}
	bool operator<=(const BigNum& b) const {return *this<b || *this==b;}
	bool operator>(const BigNum& b) const {return b<=*this;}
	bool operator>=(const BigNum& b) const {return b<=*this;}
	bool operator!=(const BigNum& b) const {return !(*this==b);}
	BigNum operator-() const {
		BigNum res = *this;
		if( res.len>0 ) res.sign*=-1;
		return res;
	}
	template<class T> BigNum operator+(const T& b) const {return *this+BigNum(b);}
	template<class T> BigNum operator-(const T& b) const {return *this-BigNum(b);}
	template<class T> bool operator==(const T& b) const {return *this==BigNum(b);}
	void print(const char *str="") const {
		if( len==0 ) printf("0");
		else {
			printf("%d", arr[len-1]*sign);
			for(int i=len-2; i>=0; --i) printf("%04d", arr[i]);
		}
		printf("%s", str);
	}
	bool isInt() const {
		if( len>2 ) return false;
		if( len<2 ) return true;
		long long res=toInt();
		return res<(1ll<<31) && res>=-(1ll<<31);
	}
	long long toInt() const {return sign*(1ll*arr[1]*step+arr[0]);}
private:
	static const int length = 10000;
	static const int digit = 4, step = 10000;
	int arr[length];
	int len, sign;
};
\end{lstlisting}

%%%%%%%%%%%%%%%%%%%%%%%%%%%%%%

\newpage
\section{Graph}
\subsection{maximum matching in general graph}
\begin{lstlisting}[label=maximum matching in general graph]
//Problem:http://acm.timus.ru/problem.aspx?space=1&num=1099
#include <cstdio>
#include <cstdlib>
#include <cstring>
#include <iostream>
#include <algorithm>
using namespace std;
const int N=250;
int n;
int head;
int tail;
int Start;
int Finish;
int link[N];		//表示哪個點匹配了哪個點
int Father[N];		//這個就是增廣路的Father……但是用起來太精髓了
int Base[N];		//該點屬於哪朵花
int Q[N];
bool mark[N];
bool map[N][N];
bool InBlossom[N];
bool in_Queue[N];
 
void CreateGraph(){
	int x,y;
	scanf("%d",&n);
	while (scanf("%d%d",&x,&y)!=EOF)
		map[x][y]=map[y][x]=1;
}
void BlossomContract(int x,int y){
	fill(mark,mark+n+1,false);
	fill(InBlossom,InBlossom+n+1,false);
	#define pre Father[link[i]]
	int lca,i;
	for (i=x;i;i=pre) {i=Base[i]; mark[i]=true; }
	for (i=y;i;i=pre) {i=Base[i]; if (mark[i]) {lca=i; break;} }  //尋找lca之旅……一定要注意i=Base[i]
	for (i=x;Base[i]!=lca;i=pre){
		if (Base[pre]!=lca) Father[pre]=link[i]; //對於BFS樹中的父邊是匹配邊的點,Father向後跳
		InBlossom[Base[i]]=true;
		InBlossom[Base[link[i]]]=true;
	}
	for (i=y;Base[i]!=lca;i=pre){
		if (Base[pre]!=lca) Father[pre]=link[i]; //同理
		InBlossom[Base[i]]=true;
		InBlossom[Base[link[i]]]=true;
	}
	#undef pre
	if (Base[x]!=lca) Father[x]=y;	 //注意不能從lca這個奇環的關鍵點跳回來
	if (Base[y]!=lca) Father[y]=x;
	for (i=1;i<=n;i++)
	  if (InBlossom[Base[i]]){
			Base[i]=lca;
			if (!in_Queue[i]){
				Q[++tail]=i;
				in_Queue[i]=true;	 //要注意如果本來連向BFS樹中父結點的邊是非匹配邊的點,可能是沒有入隊的
		  }
	  }
}
void Change(){
	int x,y,z;
	z=Finish;
	while (z){
		y=Father[z];
		x=link[y];
		link[y]=z;
		link[z]=y;
		z=x;
	}
}
void FindAugmentPath(){
	fill(Father,Father+n+1,0);
	fill(in_Queue,in_Queue+n+1,false);
	for (int i=1;i<=n;i++) Base[i]=i;
	head=0; tail=1;
	Q[1]=Start;
	in_Queue[Start]=1;
	while (head!=tail){
		int x=Q[++head];
		for (int y=1;y<=n;y++)
			if (map[x][y] && Base[x]!=Base[y] && link[x]!=y)   //無意義的邊
				if ( Start==y || link[y] && Father[link[y]] )	//精髓地用Father表示該點是否
					BlossomContract(x,y);
				else if (!Father[y]){
					Father[y]=x;
					if (link[y]){
						Q[++tail]=link[y];
						in_Queue[link[y]]=true;
					}
					else{
						Finish=y;
						Change();
						return;
					}
				}
	}
}
void Edmonds(){
	memset(link,0,sizeof(link));
	for (Start=1;Start<=n;Start++)
		if (link[Start]==0)
			FindAugmentPath();
}
void output(){
	fill(mark,mark+n+1,false);
	int cnt=0;
	for (int i=1;i<=n;i++)
		if (link[i]) cnt++;
	printf("%d\n",cnt);
	for (int i=1;i<=n;i++)
		if (!mark[i] && link[i]){
			mark[i]=true;
			mark[link[i]]=true;
			printf("%d %d\n",i,link[i]);
		}
}
int main(){
	CreateGraph();
	Edmonds();
	output();
	return 0;
}
\end{lstlisting}

%%%%%%%%%%%%%%%%%%%%%%%%%%%%%%
\newpage
\section{String}
\subsection{KMP}
\begin{lstlisting}[label=KMP]
int KMP(char ts[5005], char ss[5005]) {
	if( strlen(ts)>strlen(ss) ) return -1;
	int failure[5005];
	int len=strlen(ts);
	for(int i=1, j=failure[0]=-1; i<len; ++i) {
		while( j>=0 && ts[j+1]^ts[i] ) j=failure[j];
		if( ts[j+1]==ts[i] ) ++j;
		failure[i]=j;
	}
	for(int i=0, j=-1; ss[i]; ++i) {
		if( j>=0 && ss[i]^ts[j+1] ) j=failure[j];
		if( ss[i]==ts[j+1] ) ++j;
		if( j==len-1 ) {
			return i-len+1; // rec this!!
			j=failure[j];
		}
	}
	return -1;
}
\end{lstlisting}

\subsection{Z Algorithm}
\begin{lstlisting}[label=Z Algorithm]
void Z(char G[], int z[]){
	int len = strlen(G);
	z[0] = len;
	int L = 0, R = 1;
	for ( int i = 1 ; i < len ; i++ ) {
		if ( i >= R || z[i-L] >= R-i ) {
			int x = (i>=R) ? i : R;
			while ( x < len && G[x] == G[x-i] )  
				x++;
			z[i] = x - i;
			if ( x > i )  L = i , R = x;	
		}		
		else z[i] = z[i-L];
	}
}
\end{lstlisting}

\newpage
\subsection{Suffix Array}
\begin{lstlisting}[label=Suffix Array]
int rank[LEN], sa[LEN];
int height[LEN];
int y[LEN], cnt[LEN], rr[2][LEN];
inline bool same(int *rank, int a, int b, int l) { return rank[a]==rank[b]&&rank[a+l]==rank[b+l]; }
void sa2(char str[], int n, int m) {
	printf("%s!! %d %d\n", str, n, m);
	int *rank1=rr[0], *rank2=rr[1];
	MSET(rr[1], 0);
	int i, p;
	for(i=0; i<m; ++i) cnt[i]=0;
	for(i=0; i<n; ++i) rank2[i]=str[i], cnt[rank2[i]]++;
	for(i=1; i<m; ++i) cnt[i]+=cnt[i-1];
	for(i=n-1; i>=0; --i) sa[--cnt[rank2[i]]]=i;
	for(int j=1; p<n; j<<=1, m=p) {
		// 表示用第二個key(rank2)排序後 從 y[i] 開始的後綴排第i名
		for(p=0, i=n-j; i<n; ++i) y[p++]=i;
		for(i=0; i<n; ++i) if( sa[i]>=j ) y[p++]=sa[i]-j;
		for(i=0; i<m; ++i) cnt[i]=0;
		for(i=0; i<n; ++i) cnt[ rank2[y[i]] ] ++;
		for(i=1; i<m; ++i) cnt[i]+=cnt[i-1];
		for(i=n-1; i>=0; --i) sa[ --cnt[ rank2[y[i]] ] ]=y[i];
		for(p=i=1, rank1[sa[0]]=0; i<n; ++i)
			rank1[sa[i]]=same(rank2, sa[i], sa[i-1], j)?p-1:p++;
		std::swap(rank1, rank2);
	}
	for(int i=0; i<n; ++i) rank[i]=rank2[i];
}
void make_height(char str[]) {
	int len=strlen(str);
	height[0]=0;
	for(int i=0, j=0; i<len; ++i, j=height[rank[i-1]]-1) {
		if( rank[i]==1 ) continue;
		if( j<0 ) j=0;
		while( i+j<len && sa[rank[i]-1]+j<len &&
			str[i+j]==str[sa[rank[i]-1]+j] ) ++j;
		height[rank[i]]=j;
	}
}
int main() {
	char str[LEN];
	scanf("%s", str);
	int len = strlen(str);
	sa2(str, len+1, 256);
	make_height(str);
	for(int i=1; i<=len; ++i) printf("%d %d %s\n", i, height[i], str+sa[i]);
}
\end{lstlisting}

\newpage
\subsection{Longest Palindromic Substring}
\begin{lstlisting}[label=Longest Palindromic Substring]
char t[1001];		// 要處理的字串
cahr s[1001 * 2];	// 中間插入特殊字元的t。
int Z[1001 * 2], L, R;	// Gusfield's Algorithm
// 由a往左、由b往右,對稱地作字元比對。
int match(int a, int b) {
	int i = 0;
	while (a-i>=0 && b+i<N && s[a-i] == s[b+i]) i++;
	return i;
}
void longest_palindromic_substring()
{
	int N = strlen(t);
	// 在t中插入特殊字元,存放到s。
	memset(s, '.', N*2+1);
	for (int i=0; i<N; ++i) s[i*2+1] = t[i];
	N = N*2+1;
	// modified Gusfield's lgorithm
	Z[0] = 1;
	L = R = 0;
	for (int i=1; i<N; ++i) {
		int ii = L - (i - L);   // i的映射位置
		int n = R + 1 - i;
		if (i > R) {
			Z[i] = match(i, i);
			L = i;
			R = i + Z[i] - 1;
		}
		else if (Z[ii] == n) {
			Z[i] = n + match(i-n, i+n);
			L = i;
			R = i + Z[i] - 1;
		}
		else Z[i] = min(Z[ii], n);
	}
	// 尋找最長迴文子字串的長度。
	int n = 0, p = 0;
	for (int i=0; i<N; ++i)
		if (Z[i] > n) n = Z[p = i];
	// 記得去掉特殊字元。
	cout << "最長迴文子字串的長度是" << (n-1) / 2;
	// 印出最長迴文子字串,記得別印特殊字元。
	for (int i=p-Z[p]+1; i<=p+Z[p]-1; ++i)
		if (i & 1) cout << s[i];
}
\end{lstlisting}

%%%%%%%%%%%%%%%%%%%%%%%%%%%%%%
\newpage
\section{Math}
\subsection{Euler's phi function O(n)}
\begin{enumerate}[1.]
\item $gcd(x,y)=d \Rightarrow \phi(xy) = \frac{\phi(x) \phi(y)}{\phi(d)}$
\item $p\; is\; prime \Rightarrow \phi(p^k) = p^{k-1} \phi(p)$
\item $p\; is\; prime \Rightarrow \phi(p^k) = \phi(p^{k-1}) \times p$
\item $n = p_{1}^{k_1} p_{2}^{k_2} \cdots p_{m}^{k_m}\\
\Rightarrow \phi(n) = p_{1}^{k_1-1}\phi(p_1)\; p_{2}^{k_2-1}\phi(p_2) \cdots p_{m}^{k_m-1}\phi(p_m)$
\end{enumerate}

\begin{lstlisting}[label=Euler's phi function O(n)]
const int MAXN = 100000;
int phi[MAXN], prime[MAXN], pn=0;
memset(phi, 0, sizeof(phi));
for(int i=2; i<MAXN; ++i) {
	if( phi[i]==0 ) prime[pn++]=i, phi[i]=i-1;
	for(int j=0; j<pn; ++j) {
		if( i*prime[j]>=MAXN ) break;
		if( i%prime[j]==0 ) {
			phi[i*prime[j]] = phi[i] * prime[j];
			break;
		}
		phi[i*prime[j]] = phi[i] * phi[prime[j]];
	}
}
\end{lstlisting}

\subsection{Extended Euclid’s Algorithm}
$ ax+by=gcd(a,b) $
\begin{lstlisting}[label=Extended Euclid’s Algorithm]
int ext_gcd(int a, int b, int &x, int &y){
	int x2;
	if( b==0 ) {
		x=1, y=0;
		return a;
	}
	int gcdn=ext_gcd(b, a%b, x, y), x2=x;
	x=y, y=x2-a/b*y;
	return gcdn;
}
int ext_gcd(int a, int b, int &x, int &y){
	int t, px=1, py=0, tx,ty;
	x=0, y=1;
	while(a%b!=0) {
		tx=x, ty=y;
		x=x*(-a/b)+px, y=y*(-a/b)+py;
		px=tx, py=ty;
		t=a, a=b, b=t%b;
	}
	return b;
}
\end{lstlisting}

\subsection{Möbius function}
\begin{lstlisting}[label=Möbius function]
memset(mobius, 0, sizeof(mobius));
mobius[1]=1;
for(int i=0; i<flag; ++i) mobius[prime[i]]=-1;
for(int i=2, tmp, cntprime; i<MAXN; ++i)
{
	if( !~mobius[i] ) continue;
	tmp=i, cntprime=0;
	for(int j=0; !mobius[tmp]&&prime[j]<=tmp; ++j){
		if( tmp%prime[j]==0 )
			++cntprime, tmp/=prime[j];
		if( tmp%prime[j]==0 ) {cntprime=0;break;}
	}
	if( cntprime && mobius[tmp] )
		mobius[i]=mobius[tmp]*(cntprime&1?-1:1);
}
\end{lstlisting}

\subsection{China remainder theorem}
$ ans \equiv  a_i\; (mod\; m_i) $
\begin{lstlisting}[label=China remainder theorem]
int ans, gcdn, x, y, reduce, tmp;
for(int i=1; i<n; ++i) {
	gcdn=ext_gcd(mi[i-1], mi[i], x, y);
	reduce=ai[i]-ai[i-1];
	if( reduce%gcdn!=0 ) {
		ans=-1;
		break;
	}
	tmp=mi[i]/gcdn;
	x=(reduce/gcdn*x%tmp+tmp)%tmp;
	ai[i] = ai[i-1] + mi[i-1]*x;
	mi[i] = mi[i-1]*tmp;
}
\end{lstlisting}

%%%%%%%%%%%%%%%%%%%%%%%%%%%%%%

\newpage
\section{Others}
\subsection{8 puzzle - IDA*}
\begin{lstlisting}[label=8 puzzle - IDA*]
// 一個盤面。其數值1~8代表方塊號碼,0代表空格。
int board[3][3] = {2, 3, 4, 1, 5, 0, 7, 6, 8};
// 檢查 permutation inversion。檢查不通過,表示盤面不合理。
bool check_permutation_inversion(int board[3][3])
{
	int inversion = 0;
	for (int a=0; a<9; ++a)
		for (int b=0; b<a; ++b) {
			int i = a / 3, j = a % 3;
			int ii = b / 3, jj = b % 3;
			if (board[i][j] && board[ii][jj]
				&& board[i][j] < board[ii][jj])
				inversion++;
		}
	int row_number_of_0 = 0;
	for (int i=0; i<3 && !row_number_of_0; ++i)
		for (int j=0; j<3 && !row_number_of_0; ++j)
			if (board[i][j] == 0)
				row_number_of_0 = i+1;
	return (inversion + row_number_of_0) % 2 == 0;
}
//////////////////////////
// heuristic function,採用不在正確位置上的方塊個數。
int h(int board[3][3])
{
	int cost = 0;
	for (int i=0; i<3; ++i)
		for (int j=0; j<3; ++j)
			if (board[i][j])
				if (board[i][j] != i*3 + j + 1)
					cost++;
	return cost;
}
//////////////////////////
int taxicab_distance(int x1, int y1, int x2, int y2)
{return abs(x1 - x2) + abs(y1 - y2);}
 
// heuristic function,採用taxicab distance。
int h(int board[3][3]) {
	// 每塊方塊的正確位置。{0,0}是為了方便編寫程式而多加的。
	static const int right_pos[9][2] = {
		{0,0},
		{0,0}, {0,1}, {0,2},
		{1,0}, {1,1}, {1,2},
		{2,0}, {2,1}
	};
	// 計算每個方塊與其正確位置的 taxicab distance 的總和。
	int cost = 0;
	for (int i=0; i<3; ++i)
		for (int j=0; j<3; ++j)
			if (board[i][j])
				cost += taxicab_distance(
							i, j,
							right_pos[board[i][j]][0],
							right_pos[board[i][j]][1]
						);
	return cost;
}

// 上下左右
const string operator[4] = {"up", "down", "right", "left"};
const int dx[4] = {-1, 1, 0, 0}, dy[4] = {0, 0, 1, -1};
char solution[30];
	// 正確的推動方式,其數值是方向0~3。
const int reverse_dir[4] = {1, 0, 3, 2};
	// 用表格紀錄每一個方向的反方向。可用於避免來回推動的判斷。

int board[3][3] = {2, 3, 4, 1, 5, 0, 7, 6, 8};
	// 起始狀態。其數值1~8代表方塊號碼,0代表空格。

int sx = 1, sy = 2;
	// 空格的位置。可馬上知道推動方塊的目的地。

bool onboard(int x, int y)
{return x>=0 && x<3 && y>=0 && y<3;}
 
int IDAstar(int x, int y, int gv, int prev_dir, int& bound, bool& ans) {
	int hv = h(board);
	if (gv + hv > bound) return gv + hv;
		// 超過,回傳下次的bound
	if (hv == 0) {ans = true; return gv;}
		// 找到最佳解
 
	int next_bound = 1e9;
	for (int i=0; i<4; ++i) {
		// 四種推動方向
		int nx = x + dx[i], ny = y + dy[i];
			// 空格的新位置
		if (reverse_dir[i] == prev_dir) continue;
			// 避免來回推動
		if (!onboard(nx, ny)) continue;
			// 避免出界
		solution[gv] = oper[i];
			// 紀錄推動方向
		swap(board[x][y], board[nx][ny]);
			// 推動
		int v = IDAstar(nx, ny, gv+1, i, bound, ans);
		if (ans) return v;
		next_bound = min(next_bound, v);
		swap(board[nx][ny], board[x][y]);
			// 回復原狀態
	}
	return next_bound;
}
 
void eight_puzzle() {
	if (!check_permutation_inversion(board)) {
		cout << "盤面不合理,無法解得答案。" << endl;
		return;
	}
	// IDA*
	bool ans = false;
	int bound = 0;
	while (!ans && bound <= 50)
		bound = IDAstar(sx, sy, 0, -1, bound, ans);
	if (!ans) {
		cout << "50步內無法解得答案。" << endl;
		return;
	}
	// 印出移動方法
	for (int i=0; i<bound; ++i)
		cout << operation[solution[i]] << ' ';
	cout << endl;
}
 

\end{lstlisting}

\section*{The End}
\end{document}
